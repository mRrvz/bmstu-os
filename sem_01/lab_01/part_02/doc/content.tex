\chapter{Функции обработчика прерывания от системного таймера в защищённом режиме}

\section{UNIX-системы}

Обработчик прерывания от системного таймера \textbf{по тику} выполняет задачи:

\begin{itemize}
    \item инкремент счётчика тиков аппаратного таймера;
    \item декремент кванта текущего потока;
    \item обновление статистики использования процессора текущим процессом -- инкремент поля \textbf{c\_cpu} дескриптора текущего процесса до максимального значения 127.
    \item инкремент часов и других таймеров системы;
    \item декремент счетчика времени до отправления на выполнение отложенных вызовов, при достижении счетчиком нуля выставление флага для обработчика отложенных вызовов.
\end{itemize}

Обработчик прерывания от системного таймера \textbf{по главному тику} выполняет задачи:

\begin{itemize}
    \item регистрирует отложенные вызовы функций, относящиеся к работе планировщика, такие как пересчет приоритетов;
    \item пробуждение в нужные моменты системных процессов \textbf{swapper} и \textbf{pagedaemon}. Пробуждение обозначает регистрацию отложенного вызова процедуры \textbf{wakeup}, которая перемещает дескрипторы процессов из списка <<спящих>> в очередь готовых к выполнению.
    \item декремент счётчика времени, отвечающий за оставшееся время до посылки одного из сигналов:
        \begin{itemize}
            \item \textbf{SIGALRM} -- сигнал, посылаемый процессу по истечении времени, заданного функцией \textbf{alarm()};
            \item \textbf{SIGPROF} -- сигнал, посылаемый процессу по истечении времени заданного в таймере профилирования;
            \item \textbf{SIGVTALRM} --  сигнал, посылаемый процессу по истечении времени, заданного в <<виртуальном>> таймере.
        \end{itemize}
\end{itemize}

Обработчик прерывания от системного таймера \textbf{по кванту} выполняет задачи:

\begin{itemize}
    \item послать сигнал текущему процессу сигнал \textbf{SIGXCPU}, если тот превысил выделенную для него квоту использования процессора.
\end{itemize}


\section{Windows-системы}

Обработчик прерывания от системного таймера \textbf{по тику} выполняет задачи:

\begin{itemize}
    \item инкремент счётчика системного времени;
    \item декремент кванта текущего потока на величину, равную количеству тактов процессора, произошедших за тик. В случае, если количество затраченных потомков тактов процессора достигает квантовой цели, запускается обработка истечения кванта;
    \item декремент счетчиков времени отложенных задач;
    \item в случае, если активен механизм профилирования ядра, инициализирует отложенный вызов обработчика ловушки профилирования ядра путём постановки объекта в очередь \textbf{DPC}: обработчик ловушки профилирования регистрирует адрес команды, выполнявшейся на момент прерывания.
\end{itemize}

Обработчик прерывания от системного таймера \textbf{по главному тику} выполняет задачи:

\begin{itemize}
    \item инициализация диспетчера настройки баланса путем сбрасывания объекта <<событие>>, на котором он ожидает.
\end{itemize}

Обработчик прерывания от системного таймера \textbf{кванту} выполняет задачи:

\begin{itemize}
    \item инициализация диспетчеризации потоков -- постановка соответствующего объекта в очередь \textbf{DPC}.
\end{itemize}

\chapter{Пересчёт динамических приоритетов}

В ОС семейства \textbf{UNIX} и \textbf{UNIX} динамически пересчитываться  могут только приоритеты пользовательских процессов.

\section{UNIX-системы}

В современных системах \textbf{UNIX} ядро является вытесняющим. Это значит, что процесс в режиме ядра может быть вытеснен боле приоритетным процессом, находящимся так же в режиме ядра. Это было сделано для того, чтобы система могла обслуживать процессы реального времени, например видео или аудио.

""\newline
\noindent Согласно приоритетам процессов и принципу вытесняющего циклического планирования формируется очередь готовых к выполнению процессов. В первую очередь выполняются процессы с большим приоритетом. Процессы с одинаковыми приоритетами выполняются в течении кванта времени -- циклически, друг за другом. В случае, если процесс, имеющий более высокий приоритет поступает в очередь готовых к выполнению процессов, планировщик вытесняет текущий процесс и предоставляет ресурс более приоритетному процессу.

""\newline
\noindent Приоритет -- это целое число, находящееся в диапазоне от 0 до 127. Чем меньше значение, тем выше приоритет процесса. Приоритеты ядра варьируются от 0 до 49, а приоритеты прикладных задач от 50 до 127. Приоритеты ядра являются фиксированными величинами, а приоритеты прикладных задач могут изменяться во времени в зависимости от следующих факторов:

\begin{itemize}
    \item фактор любезности;
    \item последней измеренной величины использования процессора.
\end{itemize}

""\newline
\noindent Фактор любезности -- это целое число в диапазоне от 0 до 39 (по умолчанию 20). Чем меньше значение фактора любезности процесса, тем выше приоритет процесса. Фактор любезности процесса может быть изменен с помощью системного вызова \textbf{nice}, но только суперпользователем. Фоновым процессам задаются более высокие значения фактора любезности.

""\newline
\noindent Дескриптор процесса \textbf{proc} содержит следующие поля, которые относятся к приоритету процесса:

\begin{itemize}
    \item \textbf{p\_pri} -- текущий приоритет планирования;
    \item \textbf{p\_usrpri} -- приоритет процесса в режиме задачи;
    \item \textbf{p\_cpu} -- результат последнего измерения степени загруженности процессора (процессом);
    \item \textbf{p\_nice} -- фактор любезности, устанавливаемый пользователем.
\end{itemize}

\noindent Когда процесс находится в режиме задачи, значения \textbf{p\_pri} и \textbf{p\_usrpri} равны. Значение текущего приоритета \textbf{p\_pri} может быть повышено планировщиком для выполнения процесса в режиме ядра, а \textbf{p\_usrpri} будет использоваться для хранения приоритета который будет назначение когда процесс вернется в режим задачи.

""\newline 
\noindent Ядро системы связывает приоритет сна с событие или ожидаемым ресурсом, из-за которого процесс может блокироваться. В тот момент когда процесс просыпается, после того как был блокирован в системном вызове, ядро устанавливается приоритет сна в поле \textbf{p\_pri} -- это значение приоритета в диапазоне от 0 до 49, зависящее от события или ресурса по которому произошла блокировка. В таблице \ref{tab:bsd} приведены значения приоритетов сна для систем \textbf{4.3BSD}.


\begin{table}[h]
    \caption{Таблица приоритетов в системе \textbf{4.3BSD}}
    \label{tab:bsd}
    \begin{center}
        \begin{tabular}{ |c|c|c|  }
            \hline
            \textbf{Приоритет} & \textbf{Значение} & \textbf{Описание} \\
            \hline
            \texttt{PSWP} & 0 & Свопинг \\
            \hline
            \texttt{PSWP + 1} & 1 & Страничный демон \\
            \hline
            \texttt{PSWP + 1/2/4} & 1/2/4 & Другие действия по обработке памяти \\
            \hline
            \texttt{PINOD} & 10 & Ожидание освобождения inode \\
            \hline
            \texttt{PRIBIO} & 20 & Ожидание дискового ввода-вывода \\
            \hline
            \texttt{PRIBIO + 1} & 21 & Ожидание освобождения буфера \\
            \hline
            \texttt{PZERO} & 25 & Базовый приоритет \\
            \hline
            \texttt{TTIPRI} & 28 & Ожидание ввода с терминала \\
            \hline
            \texttt{TTOPRI} & 29 & Ожидание вывода с терминала \\
            \hline 
            \texttt{PWAIT} & 30 & Ожидание завершения процесса потомка \\
            \hline
            \texttt{PLOCK} & 35 & Консультативное ожидание блокированного ресурса \\
            \hline
            \texttt{PSLEP} & 40 & Ожидание сигнала \\
            \hline
        \end{tabular}
    \end{center}
\end{table}

""\newline 
\noindent При создании процесса после \textbf{p\_cpu} инициализируется нулём. На каждом тике обработчик таймера увеличивает это поле для текущего процесса на единицу, до максимального значения, которое равно 127. Каждую секунду обработчик прерывания инициализирует отложенный вызов процедуры \textbf{schedcpy()}, которая уменьшает значение \textbf{p\_cpu} каждого процесса исходя из фактора <<полураспада>>. В системе \textbf{4.3BSD} фактор полураспада рассчитывается по формуле \eqref{for:bsd}: 

\begin{equation}
    \label{for:bsd}
    decay = \frac{2 \cdot load\_average}{2 \cdot load\_average + 1}
\end{equation}
где \textbf{load\_average} -- среднее количество процессов, находящихся в состоянии готовности к выполнению (за последнюю секунду).

""\newline
\noindent Приоритеты для режима задачи всех процессов в процедуре \textbf{schedcpy()} пересчитываются по формуле \eqref{for:sched}:

\begin{equation}
    \label{for:sched}
    p\_usrpri = PUSER + \frac{p\_cpu}{2} + 2 \cdot p\_nice
\end{equation}
где \textbf{PUSER} -- базовый приоритет в режиме задачи, который равен 50.

""\newline 
\noindent Если процесс в последний раз использовал большое количество процессорного времени, его \textbf{p\_cpu} будет увеличен. Это приведёт к росту значения \textbf{p\_usrpri}, что приведет к понижению приоритету. Чем дольше процесс простаивает в очереди на исполнение, тем больше фактор полураспад уменьшает его \textbf{p\_cpu}, что приводит к повышению его приоритета. Данная схема предотвращает зависание низкоприоритетных по вине операционной системы. Применение такой схемы предпочтительнее процессам, которые осуществляют много операций ввода-вывода, в противоположность процессам, производящим много вычислений.

\section{Windows-системы}

\noindent В Windows при создании процесса, для него назначается базовый приоритет. Относительно базового приоритета процесса потоку назначается приоритет.

""\newline 
\noindent Планирование осуществляется на основе приоритетов потоков, готовых к выполнению. Поток с более низким приоритетом вытесняется потоком с более высоким приоритетом, в тот момент когда этот поток становится готовым к выполнению. По истечению кванта времени текущего потока, ресурс передается самому приоритетному потоку в очереди готовых на выполнение.
gs
""\newline
\noindent Раз в секунду диспетчер настройки баланса сканирует очередь готовых потоков, и, в случае, если обнаружены потоки, ожидающие выполнения более 4 секунд, диспетчер настройки баланса повышает их приоритет до 15. Как только квант истекает, приоритет потока снижается до базового приоритета. В случае, если поток не был завершен за квант времени или был вытеснен потоком с более высоким приоритетом, то после снижения приоритета поток возвращается в очередь готовых потоков.

""\newline 
\noindent Для того чтобы минимизировать расход процессорного времени, диспетчер настройки баланса сканирует только 16 готовых потоков. Диспетчер повышает приоритет не более чем у 10 потоков за один проход. Если он обнаруживает 10 потоков, приоритет которых следует повысить, он прекращает сканирование. При следующем проходе сканирование возобновляется с того места, где оно было прервано. Наличие 10 потоков, приоритет которых нужно повысить, говорит о высокой загруженности системы.

""\newline 
\noindent Windows использует 32 уровня приоритета, которые описываются целыми числами от 0 до 31, а 31 -- наивысший приоритет. Приоритеты от 16 до 31 -- уровни реального времени, от 0 до 15 -- динамические уровни. Уровень 0 зарезервирован для потока обнуления страниц.

""\newline
\noindent Уровни приоритета потоков назначаются с двух позиций: \textbf{Windows API} и ядра операционной системы.\textbf{Windows API} сортирует процессы по классам приоритета, которые были назначены при их создании:

\begin{itemize}
    \item реального времени (real-time, 4);
    \item высокий (high, 3);
    \item выше обычного (above normal, 6);
    \item обычный (normal, 2);
    \item ниже обычного (below normal, 5).
    \item простой (idle, 1).
\end{itemize}

\noindent API-функция SetPriorityClass позволяет изменять класс приоритета процесса до одного из этих уровней.

""\newline
\noindent Затем назначается относительный приоритет потоков в рамках процессов:

\begin{itemize} 
    \item критичный по времени (time critical, 15);
    \item наивысший (highest, 2);
    \item выше обычного (above normal, 1);
    \item обычный (normal, 0);
    \item ниже обычного (below normal, -1);
    \item низший (lowest, -2);
    \item простой (idle, -15).
\end{itemize} 

\noindent Исходный базовый приоритет потока наследуется от базового приоритета процесса. Процесс по умолчанию наследует свой базовый приоритет у того процесса, который его создал. 

""\newline
\noindent Таким образом, в \textbf{Windows API} каждый поток имеет базовый приоритет, являю­щийся функцией класса приоритета процесса и его относительного приоритета процесса. В  ядре класс приоритета процесса преобразуется в базовый приоритет. В таблице \ref{tbl:priority} приведено соответствие между приоритетами \textbf{Windows API} и ядра системы приоритета.

\begin{table}[h]
    \caption{Соответствие между приоритетами \textbf{Windows API} и ядра Windows}
    \begin{center}
        \begin{tabular}{|l|p{45pt}|p{45pt}|p{45pt}|p{45pt}|p{45pt}|p{45pt}|}
            \hline
            {} & \textbf{real-time} & \textbf{high} & \textbf{above normal} & \textbf{normal} & \textbf{below normal} & \textbf{idle}\\
            \hline
            \textbf{time critical} & 31 & 15 & 15 & 15 & 15 & 15 \\
            \hline
            \textbf{highest} & 26 & 15 & 12 & 10 & 8 & 6 \\
            \hline
            \textbf{above normal} & 25 & 14 & 11 & 9 & 7 & 5 \\
            \hline
            \textbf{normal} & 24 & 13 & 10 & 8 & 6 & 4 \\
            \hline
            \textbf{below normal} & 23 & 12 & 9 & 7 & 5 & 3 \\
            \hline
            \textbf{lowest} & 22 & 11 & 8 & 6 & 4 & 2 \\
            \hline
            \textbf{idle} & 16 & 1 & 1 & 1 & 1 & 1 \\
            \hline
        \end{tabular}
    \end{center}
    \label{tbl:priority}
\end{table}


\noindent Текущий приоритет потока в динамическом диапазоне (от 1 до 15) может быть изменён планировщиком вследствие причин:

\begin{itemize}
    \item повышение приоритета после завершения операций ввода-вывода;
    \item повышение приоритета владельца блокировки;
    \item повышение приоритета вследствие ввода из пользовательского интерфейса;
    \item повышение приоритета вследствие длительного ожидания ресурса исполняющей системы;
    \item повышение приоритета вследствие ожидания объекта ядра;
    \item повышение приоритета в случае, когда готовый к выполнению поток не был запущен в течение длительного времени;
    \item повышение приоритета проигрывания мультимедиа службой планировщика \textbf{MMCSS}.
\end{itemize}

""\newline 
\noindent Текущий приоритет потока в динамическом диапазоне может быть понижен до базового путем вычитания всех его повышений. В таблице \ref{tab:io} приведены рекомендуемые значения повышения приоритета для устройств ввода-вывода.

\begin{table}[h!]
    \caption{Рекомендуемые значения повышения приоритета.}
    \begin{center}
        \begin{tabular}{|p{100mm}|l|}
            \hline
            \textbf{Устройство} & \textbf{Приращение} \\
            \hline
            Диск, CD-ROM, параллельный порт, видео & 1 \\
            \hline
            Сеть, почтовый ящик, именованный канал, последовательный порт & 2 \\
            \hline
            Клавиатура, мышь & 6 \\
            \hline
            Звуковая плата & 8 \\
            \hline
        \end{tabular}
    \end{center}
    \label{tab:io}
\end{table}

\subsection{MMCSS}

Потоки, на которых выполняются различные мультимедийные приложения, должны выполняться с минимальными задержками. В Windows эта задача решается путем повышения приоритетов таких потоков драйвером \textbf{MMCSS} -- MultiMedia Class Scheduler Service. Приложения, которые реализуют воспроизведение мультимедиа, указывают драйверу \textbf{MMCSS} задачу из списка:

\begin{itemize}
	\item аудио;
	\item игры;
	\item распределение;
	\item захват;
	\item воспроизведение;
	\item задачи администратора многоэкранного режима.
\end{itemize}

Одно из наиболее важных свойств для планирования потоков -- категория планирования -- первичный фактор определяющий приоритет потоков, зарегистрированных в \textbf{MMCSS}. Различные категории планирования представленны в таблице \ref{tab:plan}.

\begin{table}[h]
	\caption{Категории планирования.}
	\begin{center}
		\begin{tabular}{|p{40mm}|p{30mm}|p{80mm}|}
			\hline
			\textbf{Категория} & \textbf{Приоритет} & \textbf{Описание} \\
			\hline
			High (Высокая) & 23-26 & Потоки профессионального аудио (Pro Audio), запущенные с приоритетом выше, чем у других потоков на системе, за исключением критических системных потоков \\
			\hline
			Medium (Средняя) & 16-22 & Потоки, являющиеся частью приложений первого плана, например Windows Media Player \\
			\hline
			Low (Низкая) & 8-15 & Все остальные потоки, не являющиеся частью предыдущих категорий \\
			\hline
			Exhausted (Исчерпавших потоков) & 1-7 & Потоки, исчерпавшие свою долю времени центрального процессора, выполнение которых продолжиться, только если не будут готовы к выполнению другие потоки с более высоким уровнем приоритета \\
			\hline
		\end{tabular}
	\end{center}
	\label{tab:plan}
\end{table}

Функции \textbf{MMCSS} временно повышают приоритет потоков, зарегистрированных с \textbf{MMCS} до уровня, соответствующего их категориям планирования. Далее, их приоритет снижается до уровня, соответствующего категории \textbf{Exhausted}, для того чтобы другие потоки могли получить ресурс.

\section*{Вывод}

Функции обработчика прерывания от системного таймера в защищенном режиме в системах Unix и Windows выполняют одинаковые действия:

\begin{itemize} 
	\item выполняют декремент счетчиков времени: часов, таймеров, счетчиков времени отложенных действий, будильников реального времени;
	\item выполняют декремент кванта текущего процесса в Linux и декремент текущего потока в Windows;
	\item инициализируют отложенные действия, относящиеся к работе планировщика, такие как пересчёт приоритетов.
\end{itemize} 

\noindent Обе системы являются системами разделения времени с динамическими приоритетами и вытеснением. Это объясняется тем, что такой подход позволяет поддерживать процессы реального времени, такие как аудио и видео. Пересчет динамических приоритетов пользовательских процессов выполняется для того чтобы исключить их бесконечное откладывание.

%Пересчёт динамических приоритетов в этих системах можно описать так:

%\begin{itemize}
%	\item при создании процесса в Windows, ему назначается приоритет, называемый базовым. Приоритеты потоков определяются относительно приоритета процесса, в котором они создаются. Приоритет потока пользовательского процесса может быть пересчитан динамически;
%	\item в UNIX приоритет процесса характеризуют текущим приоритетом и приоритетом процесса в режиме задачи. Приоритет процесса в режиме задачи может быть динамически пересчитан в зависимости от фактора любезности процесса и величины использования процессора. Приоритеты ядра являются фиксированными величинами.
%\end{itemize}