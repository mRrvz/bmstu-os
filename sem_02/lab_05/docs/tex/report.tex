\documentclass[12pt]{report}
\usepackage[utf8]{inputenc}
\usepackage[russian]{babel}
%\usepackage[14pt]{extsizes}
\usepackage{listings}
\usepackage{graphicx}
\usepackage{amsmath,amsfonts,amssymb,amsthm,mathtools} 
\usepackage{pgfplots}
\usepackage{filecontents}
\usepackage{float}
\usepackage{comment}
\usepackage{indentfirst}
\usepackage{eucal}
\usepackage{enumitem}
%s\documentclass[openany]{book}
\frenchspacing

\usepackage{indentfirst} % Красная строка

\usetikzlibrary{datavisualization}
\usetikzlibrary{datavisualization.formats.functions}

\usepackage{amsmath}


% Для листинга кода:
\lstset{ %
	language=c,                 % выбор языка для подсветки (здесь это С)
	basicstyle=\small\sffamily, % размер и начертание шрифта для подсветки кода
	numbers=left,               % где поставить нумерацию строк (слева\справа)
	numberstyle=\tiny,           % размер шрифта для номеров строк
	stepnumber=1,                   % размер шага между двумя номерами строк
	numbersep=5pt,                % как далеко отстоят номера строк от подсвечиваемого кода
	showspaces=false,            % показывать или нет пробелы специальными отступами
	showstringspaces=false,      % показывать или нет пробелы в строках
	showtabs=false,             % показывать или нет табуляцию в строках
	frame=single,              % рисовать рамку вокруг кода
	tabsize=2,                 % размер табуляции по умолчанию равен 2 пробелам
	captionpos=t,              % позиция заголовка вверху [t] или внизу [b] 
	breaklines=true,           % автоматически переносить строки (да\нет)
	breakatwhitespace=false, % переносить строки только если есть пробел
	escapeinside={\#*}{*)}   % если нужно добавить комментарии в коде
}


\usepackage[left=2cm,right=2cm, top=2cm,bottom=2cm,bindingoffset=0cm]{geometry}
% Для измененных титулов глав:
\usepackage{titlesec, blindtext, color} % подключаем нужные пакеты
\definecolor{gray75}{gray}{0.75} % определяем цвет
\newcommand{\hsp}{\hspace{20pt}} % длина линии в 20pt
% titleformat определяет стиль
\titleformat{\chapter}[hang]{\Huge\bfseries}{\thechapter\hsp\textcolor{gray75}{|}\hsp}{0pt}{\Huge\bfseries}


% plot
\usepackage{pgfplots}
\usepackage{filecontents}
\usetikzlibrary{datavisualization}
\usetikzlibrary{datavisualization.formats.functions}

\begin{document}
	%\def\chaptername{} % убирает "Глава"
	\thispagestyle{empty}
	\begin{titlepage}
		\noindent \begin{minipage}{0.15\textwidth}
			\includegraphics[width=\linewidth]{img/b_logo}
		\end{minipage}
		\noindent\begin{minipage}{0.9\textwidth}\centering
			\textbf{Министерство науки и высшего образования Российской Федерации}\\
			\textbf{Федеральное государственное бюджетное образовательное учреждение высшего образования}\\
			\textbf{~~~«Московский государственный технический университет имени Н.Э.~Баумана}\\
			\textbf{(национальный исследовательский университет)»}\\
			\textbf{(МГТУ им. Н.Э.~Баумана)}
		\end{minipage}
		
		\noindent\rule{18cm}{3pt}
		\newline\newline
		\noindent ФАКУЛЬТЕТ $\underline{\text{«Информатика и системы управления»}}$ \newline\newline
		\noindent КАФЕДРА $\underline{\text{«Программное обеспечение ЭВМ и информационные технологии»}}$\newline\newline\newline\newline\newline
		
		\begin{center}
			\noindent\begin{minipage}{1.1\textwidth}\centering
				\Large\textbf{  Отчет по лабораторной работе №5}\newline
				\textbf{по дисциплине <<Операционные системы>>}\newline\newline\newline
			\end{minipage}
		\end{center}
		
		\noindent\textbf{Тема} $\underline{\text{Буферизованный ввод / вывод}}$\newline\newline
		\noindent\textbf{Студент} $\underline{\text{Романов А.В.~~~~~~~~~~~~~~~~~~~~}}$\newline\newline
		\noindent\textbf{Группа} $\underline{\text{ИУ7-63Б~~~~~~~~~~~~~~~~~~~~~~~~~~~~}}$\newline\newline
		\noindent\textbf{Оценка (баллы)} $\underline{\text{~~~~~~~~~~~~~~~~~~~~~~~~~~~}}$\newline\newline
		\noindent\textbf{Преподаватель} $\underline{\text{Рязанова Н. Ю.~~~~~~~}}$\newline\newline\newline
		
		\begin{center}
			\vfill
			Москва~---~\the\year
			~г.
		\end{center}
	\end{titlepage}
	

\chapter{Первая программа}

\begin{lstlisting}[language=c, label=p1, caption=Программа №1]
#include <stdio.h>
#include <fcntl.h>

#define OK 0
#define BUF_SIZE 20
#define VALID_READED 1

#define FILE_NAME "data/alphabet.txt"
#define SPEC "%c"

int main(void)
{
	int fd = open(FILE_NAME, O_RDONLY);
	
	FILE *fs1 = fdopen(fd, "r");
	char buff1[BUF_SIZE];
	setvbuf(fs1, buff1, _IOFBF, BUF_SIZE);
	
	FILE *fs2 = fdopen(fd, "r");
	char buff2[BUF_SIZE];
	setvbuf(fs2, buff2, _IOFBF, BUF_SIZE);
	
	int flag1 = 1, flag2 = 2;
	while (flag1 == VALID_READED || flag2 == VALID_READED)
	{
		char c;
		
		if ((flag1 = fscanf(fs2, SPEC, &c)) == VALID_READED)
		{
			fprintf(stdout, SPEC, c);
		}
		
		if ((flag2 = fscanf(fs2, SPEC, &c) == VALID_READED)
		{
			fprintf(stdout, SPEC, c);
		}
	}
	
	return OK;
}
\end{lstlisting}

\begin{lstlisting}[language=c, label=p1thread, caption=Программа №1 (реализация с потоками)]
#include <stdio.h>
#include <fcntl.h>
#include <pthread.h>

#define OK 0
#define BUF_SIZE 20
#define VALID_READED 1

#define FILE_NAME "data/alphabet.txt"
#define SPEC "%c"

void *run_buffer(void *args)
{
	int flag = 1;
	FILE *fs = (FILE *)args;
	
	while (flag == VALID_READED)
	{
		char c;
		if ((flag = fscanf(fs, SPEC, &c)) == VALID_READED)
		{
			fprintf(stdout, SPEC, c);
		}
	}
	
	return NULL;
}

int main(void)
{
	setbuf(stdout, NULL);
	pthread_t thread;
	int fd = open(FILE_NAME, O_RDONLY);
	
	FILE *fs1 = fdopen(fd, "r");
	char buff1[BUF_SIZE];
	setvbuf(fs1, buff1, _IOFBF, BUF_SIZE);
	
	FILE *fs2 = fdopen(fd, "r");
	char buff2[BUF_SIZE];
	setvbuf(fs2, buff2, _IOFBF, BUF_SIZE);
	
	int rc = pthread_create(&thread, NULL, run_buffer, (void *)fs2);
	
	int flag = 1;
	while (flag == VALID_READED)
	{
		char c;
		fprintf(stdout, "\nSCANF IN MAIN_1");
		flag = fscanf(fs1, SPEC, &c);
		fprintf(stdout, "\nSCANF IN MAIN_2");
		if (flag == 1)
		{
			fprintf(stdout, SPEC, c);
		}
	}
	
	pthread_join(thread, NULL);
	return OK;
}
\end{lstlisting}

\chapter{Вторая программа}

\begin{lstlisting}[language=c, label=p2, caption=Программа №2]
#include <fcntl.h>
#include <unistd.h>

#define OK 0
#define VALID_READED 1
#define FILE_NAME "data/alphabet.txt"

int main(void)
{
	int fd1 = open(FILE_NAME, O_RDONLY);
	int fd2 = open(FILE_NAME, O_RDONLY);
	int rc1, rc2 = VALID_READED;
	
	while (rc1 == VALID_READED || rc2 == VALID_READED)
	{
		char c;
		
		rc1 = read(fd1, &c, 1);
		if (rc1 == VALID_READED)
		{
			write(1, &c, 1);
		}
		
		rc2 = read(fd2, &c, 1);
		if (rc2 == VALID_READED)
		{
			write(1, &c, 1);
		}
	}
	
	return OK;
}
\end{lstlisting}

\begin{lstlisting}[language=c, label=p2thread, caption=Программа №2 (реализация с потоками)]
#include <stdio.h>
#include <fcntl.h>
#include <unistd.h>
#include <pthread.h>

#define OK 0
#define VALID_READED 1
#define FILE_NAME "data/alphabet.txt"

void *run_buffer(void *args)
{
	int fd = *((int *)args);
	int err = VALID_READED;
	
	while (err == VALID_READED)
	{
		char c;
		err = read(fd, &c, 1);
		if (err == VALID_READED)
		{
			write(1, &c, 1);
		}
	}

	return NULL;
}

int main(void)
{
	int fd1 = open(FILE_NAME, O_RDONLY);
	int fd2 = open(FILE_NAME, O_RDONLY);
	
	pthread_t thread;
	int rc = pthread_create(&thread, NULL, run_buffer, (void *)(&fd2));
	int err = VALID_READED;
	
	while (err == VALID_READED)
	{
		char c;
		err = read(fd1, &c, 1);
		if (err == VALID_READED)
		{
			write(1, &c, 1);
		}
	}
	
	pthread_join(thread, NULL);
	return OK;
}

\end{lstlisting}

\chapter{Третья программа}

\begin{lstlisting}[language=c, label=p3, caption=Программа №2]
#include <stdio.h>
#include <fcntl.h>
#include <unistd.h>

#define OK 0
#define FILE_NAME "data/out.txt"
#define SPEC "%c"

int main()
{
	FILE *f1 = fopen(FILE_NAME, "w");
	FILE *f2 = fopen(FILE_NAME, "w");
	
	for (char c = 'a'; c <= 'z'; c++)
	{
		if (c % 2)
		{
			fprintf(f1, SPEC, c);
		}
		else
		{
			fprintf(f2, SPEC, c);
		}
	}
	
	fclose(f2);
	fclose(f1);
	
	return OK;
}
\end{lstlisting}

\begin{lstlisting}[language=c, label=p3thread, caption=Программа №3 (реализация с потоками)]
#include <stdio.h>
#include <fcntl.h>
#include <pthread.h>
#include <unistd.h>

#define OK 0
#define FILE_NAME "data/out.txt"
#define SPEC "%c"

void *run_buffer(void *args)
{
	FILE *f = (FILE *)args;
	
	for (char c = 'b'; c <= 'z'; c += 2)
	{
		fprintf(f, SPEC, c);
	}
	
	fclose(f);
	return NULL;
}

int main()
{
	FILE *f1 = fopen(FILE_NAME, "w");
	FILE *f2 = fopen(FILE_NAME, "w");
	
	pthread_t thread;
	int rc = pthread_create(&thread, NULL, run_buffer, (void *)(f2));
	
	for (char c = 'a'; c <= 'z'; c += 2)
	{
		fprintf(f1, SPEC, c);
	}
	
	pthread_join(thread, NULL);
	fclose(f1);
	
	return OK;
}
\end{lstlisting}

\bibliographystyle{utf8gost705u}  % стилевой файл для оформления по ГОСТу
\bibliography{51-biblio}          % имя библиографической базы (bib-файла)
	
\end{document}
