\documentclass[12pt]{report}
\usepackage[utf8]{inputenc}
\usepackage[russian]{babel}
%\usepackage[14pt]{extsizes}
\usepackage{listings}
\usepackage{graphicx}
\usepackage{amsmath,amsfonts,amssymb,amsthm,mathtools} 
\usepackage{pgfplots}
\usepackage{filecontents}
\usepackage{float}
\usepackage{comment}
\usepackage{indentfirst}
\usepackage{eucal}
\usepackage{enumitem}
%s\documentclass[openany]{book}
\frenchspacing

\usepackage{indentfirst} % Красная строка

\usetikzlibrary{datavisualization}
\usetikzlibrary{datavisualization.formats.functions}

\usepackage{amsmath}


% Для листинга кода:
\lstset{ %
	language=c,                 % выбор языка для подсветки (здесь это С)
	basicstyle=\small\sffamily, % размер и начертание шрифта для подсветки кода
	numbers=left,               % где поставить нумерацию строк (слева\справа)
	numberstyle=\tiny,           % размер шрифта для номеров строк
	stepnumber=1,                   % размер шага между двумя номерами строк
	numbersep=5pt,                % как далеко отстоят номера строк от подсвечиваемого кода
	showspaces=false,            % показывать или нет пробелы специальными отступами
	showstringspaces=false,      % показывать или нет пробелы в строках
	showtabs=false,             % показывать или нет табуляцию в строках
	frame=single,              % рисовать рамку вокруг кода
	tabsize=2,                 % размер табуляции по умолчанию равен 2 пробелам
	captionpos=t,              % позиция заголовка вверху [t] или внизу [b] 
	breaklines=true,           % автоматически переносить строки (да\нет)
	breakatwhitespace=false, % переносить строки только если есть пробел
	escapeinside={\#*}{*)}   % если нужно добавить комментарии в коде
}


\usepackage[left=2cm,right=2cm, top=2cm,bottom=2cm,bindingoffset=0cm]{geometry}
% Для измененных титулов глав:
\usepackage{titlesec, blindtext, color} % подключаем нужные пакеты
\definecolor{gray75}{gray}{0.75} % определяем цвет
\newcommand{\hsp}{\hspace{20pt}} % длина линии в 20pt
% titleformat определяет стиль
\titleformat{\chapter}[hang]{\Huge\bfseries}{\thechapter\hsp\textcolor{gray75}{|}\hsp}{0pt}{\Huge\bfseries}


% plot
\usepackage{pgfplots}
\usepackage{filecontents}
\usetikzlibrary{datavisualization}
\usetikzlibrary{datavisualization.formats.functions}

\begin{document}
	%\def\chaptername{} % убирает "Глава"
	\thispagestyle{empty}
	\begin{titlepage}
		\noindent \begin{minipage}{0.15\textwidth}
			\includegraphics[width=\linewidth]{img/b_logo}
		\end{minipage}
		\noindent\begin{minipage}{0.9\textwidth}\centering
			\textbf{Министерство науки и высшего образования Российской Федерации}\\
			\textbf{Федеральное государственное бюджетное образовательное учреждение высшего образования}\\
			\textbf{~~~«Московский государственный технический университет имени Н.Э.~Баумана}\\
			\textbf{(национальный исследовательский университет)»}\\
			\textbf{(МГТУ им. Н.Э.~Баумана)}
		\end{minipage}
		
		\noindent\rule{18cm}{3pt}
		\newline\newline
		\noindent ФАКУЛЬТЕТ $\underline{\text{«Информатика и системы управления»}}$ \newline\newline
		\noindent КАФЕДРА $\underline{\text{«Программное обеспечение ЭВМ и информационные технологии»}}$\newline\newline\newline\newline\newline
		
		\begin{center}
			\noindent\begin{minipage}{1.1\textwidth}\centering
				\Large\textbf{  Отчет по лабораторной работе №5}\newline
				\textbf{по дисциплине <<Операционные системы>>}\newline\newline\newline
			\end{minipage}
		\end{center}
		
		\noindent\textbf{Тема} $\underline{\text{Буферизованный ввод / вывод}}$\newline\newline
		\noindent\textbf{Студент} $\underline{\text{Романов А.В.~~~~~~~~~~~~~~~~~~~~}}$\newline\newline
		\noindent\textbf{Группа} $\underline{\text{ИУ7-63Б~~~~~~~~~~~~~~~~~~~~~~~~~~~~}}$\newline\newline
		\noindent\textbf{Оценка (баллы)} $\underline{\text{~~~~~~~~~~~~~~~~~~~~~~~~~~~}}$\newline\newline
		\noindent\textbf{Преподаватель} $\underline{\text{Рязанова Н. Ю.~~~~~~~}}$\newline\newline\newline
		
		\begin{center}
			\vfill
			Москва~---~\the\year
			~г.
		\end{center}
	\end{titlepage}


\chapter{Структура \_IO\_FILE}

\begin{lstlisting}[language=c, label=_io_file, caption=Листинг структуры \_IO\_FILE]
struct _IO_FILE
{
	int _flags;    /* High-order word is _IO_MAGIC; rest is flags. */

	/* The following pointers correspond to the C++ streambuf protocol. */
	char *_IO_read_ptr;  /* Current read pointer */
	char *_IO_read_end;  /* End of get area. */
	char *_IO_read_base;  /* Start of putback+get area. */
	char *_IO_write_base;  /* Start of put area. */
	char *_IO_write_ptr;  /* Current put pointer. */
	char *_IO_write_end;  /* End of put area. */
	char *_IO_buf_base;  /* Start of reserve area. */
	char *_IO_buf_end;  /* End of reserve area. */
	
	/* The following fields are used to support backing up and undo. */
	char *_IO_save_base; /* Pointer to start of non-current get area. */
	char *_IO_backup_base;  /* Pointer to first valid character of backup area */
	char *_IO_save_end; /* Pointer to end of non-current get area. */
	
	struct _IO_marker *_markers;
	
	struct _IO_FILE *_chain;
	
	int _fileno;
	int _flags2;
	__off_t _old_offset; /* This used to be _offset but it's too small.  */
	
	/* 1+column number of pbase(); 0 is unknown. */
	unsigned short _cur_column;
	signed char _vtable_offset;
	char _shortbuf[1];
	
	_IO_lock_t *_lock;
	#ifdef _IO_USE_OLD_IO_FILE
};

struct _IO_FILE_complete
{
	struct _IO_FILE _file;
	#endif
	__off64_t _offset;
	/* Wide character stream stuff.  */
	struct _IO_codecvt *_codecvt;
	struct _IO_wide_data *_wide_data;
	struct _IO_FILE *_freeres_list;
	void *_freeres_buf;
	size_t __pad5;
	int _mode;
	/* Make sure we don't get into trouble again.  */
	char _unused2[15 * sizeof (int) - 4 * sizeof (void *) - sizeof (size_t)];
};
\end{lstlisting}
	
\chapter{Первая программа}

\textbf{На рис. \ref{fig:prog_01_schema} представлена схема структур, используемых в первой программе.}

\begin{figure}[H]
	\centering
	\includegraphics[scale=0.33]{img/prog_01_schema.jpg}
	\caption{Схема структур программы №1}
	\label{fig:prog_01_schema}
\end{figure}

\begin{itemize}
	\item Функция \texttt{open()} создает новый файловый дескриптор файла (открытого только на чтение) "alphabet.txt", запись в системной таблице открыт файлов. Эта запись регистрирует смещение в файле и флаги состояния файла;
	
	\item функция \texttt{fdopen()} создает указатели на структуру \texttt{FILE}. Поле \texttt{\_fileno} содержит дескриптор, который вернула функция \texttt{fopen()};
	
	\item функция \texttt{setvbuf()} явно задает размер буффера в 20 байт и меняет тип буферизации (для \texttt{fs1} и \texttt{fs2}) на полную;
	
	\item при первом вызове функции \texttt{fscanf()} в цикле (для \texttt{fs1}), \texttt{buff1} будет заполнен полностью -- первыми 20 символами (буквами алфавита). \texttt{f\_pos} в структуре \texttt{struct\_file} открытого файла увеличится на 20;
	
	\item при втором вызове \texttt{fscanf()} в цикле (для \texttt{fs2}) буффер \texttt{buff2} будет заполнен оставшимися 6 символами (начиная с \texttt{f\_pos});
	
	\item в цикле поочередно выводятся символы из \texttt{buff1} и \texttt{buff2}.
\end{itemize}

\begin{lstlisting}[language=c, label=p1, caption=Программа №1]
#include <stdio.h>
#include <fcntl.h>

#define OK 0
#define BUF_SIZE 20
#define VALID_READED 1

#define FILE_NAME "data/alphabet.txt"
#define SPEC "%c"

int main(void)
{
	int fd = open(FILE_NAME, O_RDONLY);
	
	FILE *fs1 = fdopen(fd, "r");
	char buff1[BUF_SIZE];
	setvbuf(fs1, buff1, _IOFBF, BUF_SIZE);
	
	FILE *fs2 = fdopen(fd, "r");
	char buff2[BUF_SIZE];
	setvbuf(fs2, buff2, _IOFBF, BUF_SIZE);
	
	int flag1 = 1, flag2 = 2;
	while (flag1 == VALID_READED || flag2 == VALID_READED)
	{
		char c;
		
		if ((flag1 = fscanf(fs2, SPEC, &c)) == VALID_READED)
		{
			fprintf(stdout, SPEC, c);
		}
		
		if ((flag2 = fscanf(fs2, SPEC, &c) == VALID_READED)
		{
			fprintf(stdout, SPEC, c);
		}
	}
	
	return OK;
}
\end{lstlisting}

""\newline
\textbf{На рис. \ref{fig:prog_01} представлен результат работы первой программы.}

\begin{figure}[H]
	\centering
	\includegraphics[scale=0.85]{img/prog_01.png}
	\caption{Результат работы первой программы}
	\label{fig:prog_01}
\end{figure}

\begin{lstlisting}[language=c, label=p1thread, caption=Программа №1 (реализация с потоками)]
#include <stdio.h>
#include <fcntl.h>
#include <pthread.h>
#define OK 0
#define BUF_SIZE 20
#define VALID_READED 1
#define FILE_NAME "data/alphabet.txt"
#define SPEC "%c"

void *run_buffer(void *args)
{
	int flag = 1;
	FILE *fs = (FILE *)args;
	while (flag == VALID_READED)
	{
		char c;
		if ((flag = fscanf(fs, SPEC, &c)) == VALID_READED)
		{
			fprintf(stdout, SPEC, c);
		}
	}
	return NULL;
}

int main(void)
{
	setbuf(stdout, NULL);
	pthread_t thread;
	int fd = open(FILE_NAME, O_RDONLY);
	
	FILE *fs1 = fdopen(fd, "r");
	char buff1[BUF_SIZE];
	setvbuf(fs1, buff1, _IOFBF, BUF_SIZE);
	
	FILE *fs2 = fdopen(fd, "r");
	char buff2[BUF_SIZE];
	setvbuf(fs2, buff2, _IOFBF, BUF_SIZE);
	
	int rc = pthread_create(&thread, NULL, run_buffer, (void *)fs2);
	
	int flag = 1;
	while (flag == VALID_READED)
	{
		char c;
		fprintf(stdout, "\nSCANF IN MAIN_1");
		flag = fscanf(fs1, SPEC, &c);
		fprintf(stdout, "\nSCANF IN MAIN_2");
		if (flag == 1)
		{
			fprintf(stdout, SPEC, c);
		}
	}
	
	pthread_join(thread, NULL);
	return OK;
}
\end{lstlisting}

""\newline
\textbf{На рис. \ref{fig:prog_01_thread} представлен результат работы первой программы (с потоками).}

\begin{figure}[H]
	\centering
	\includegraphics[scale=0.6]{img/prog_01_thread.png}
	\caption{Результат работы первой программы (с потоками)}
	\label{fig:prog_01_thread}
\end{figure}

\chapter{Вторая программа}

\textbf{На рис. \ref{fig:prog_02_schema} представлена схема структур, используемых во второй программе.}

\begin{figure}[H]
	\centering
	\includegraphics[scale=0.4]{img/prog_02_schema.jpg}
	\caption{Схема структур программы №2}
	\label{fig:prog_02_schema}
\end{figure}

\begin{itemize}
	\item Функция \texttt{open()} создает файловые дескрипторы, два раза для одного и того же файла, поэтому в программе существует две различные \texttt{struct file}, но ссылающиеся на один и тот же \texttt{struct inode};
	\item из-за того что структуры разные, посимвольная печать просто дважды выведет содержимое файла в формате <<AAbbcc...>> (в случае однопоточной реализации); 
	\item в случае многопоточной реализации, вывод второго потока начнется позже (нужно время, для создание этого потока) и символы перемешаются (см. рис. \ref{fig:prog_02_thread}).
\end{itemize}

\begin{lstlisting}[language=c, label=p2, caption=Программа №2]
#include <fcntl.h>
#include <unistd.h>

#define OK 0
#define VALID_READED 1
#define FILE_NAME "data/alphabet.txt"

int main(void)
{
	int fd1 = open(FILE_NAME, O_RDONLY);
	int fd2 = open(FILE_NAME, O_RDONLY);
	int rc1, rc2 = VALID_READED;
	while (rc1 == VALID_READED || rc2 == VALID_READED)
	{
		char c;
		rc1 = read(fd1, &c, 1);
		if (rc1 == VALID_READED)
		{
			write(1, &c, 1);
		}

		rc2 = read(fd2, &c, 1);
		if (rc2 == VALID_READED)
		{
			write(1, &c, 1);
		}
	}
	return OK;
}
\end{lstlisting}

""\newline
\textbf{На рис. \ref{fig:prog_02} представлен результат работы второй программы.}

\begin{figure}[H]
	\centering
	\includegraphics[scale=0.9]{img/prog_02.png}
	\caption{Результат работы второй программы}
	\label{fig:prog_02}
\end{figure}

\begin{lstlisting}[language=c, label=p2thread, caption=Программа №2 (реализация с потоками)]
#include <stdio.h>
#include <fcntl.h>
#include <unistd.h>
#include <pthread.h>

#define OK 0
#define VALID_READED 1
#define FILE_NAME "data/alphabet.txt"

void *run_buffer(void *args)
{
	int fd = *((int *)args);
	int err = VALID_READED;
	
	while (err == VALID_READED)
	{
		char c;
		err = read(fd, &c, 1);
		if (err == VALID_READED)
		{
			write(1, &c, 1);
		}
	}

	return NULL;
}

int main(void)
{
	int fd1 = open(FILE_NAME, O_RDONLY);
	int fd2 = open(FILE_NAME, O_RDONLY);
	
	pthread_t thread;
	int rc = pthread_create(&thread, NULL, run_buffer, (void *)(&fd2));
	int err = VALID_READED;
	
	while (err == VALID_READED)
	{
		char c;
		err = read(fd1, &c, 1);
		if (err == VALID_READED)
		{
			write(1, &c, 1);
		}
	}
	
	pthread_join(thread, NULL);
	return OK;
}
\end{lstlisting}

""\newline
\textbf{На рис. \ref{fig:prog_02_thread} представлен результат работы второй программы (с потоками).}

\begin{figure}[H]
	\centering
	\includegraphics[scale=0.8]{img/prog_02_thread.png}
	\caption{Результат работы второй программы (с потоками)}
	\label{fig:prog_02_thread}
\end{figure}

\chapter{Третья программа}

\textbf{На рис. \ref{fig:prog_03_schema} представлена схема структур, используемых в третьей программе.}

\begin{figure}[H]
	\centering
	\includegraphics[scale=0.32]{img/prog_03_schema.jpg}
	\caption{Схема структур программы №2}
	\label{fig:prog_03_schema}
\end{figure}

\begin{itemize}
	\item Файл открывается на запись два раза, с помощью функции \texttt{fopen()};
	\item функция \texttt{fprintf()} предоставляет буферизованный вывод - буфер создается без нашего вмешательства;g
	\item изначально информация пишется в буфер, а из буфера в файл если произошло одно из событий:
		\subitem буффер полон;
		\subitem вызвана функция \texttt{fclose()};
		\subitem вызвана функция \texttt{fflush()};
	\item в случае нашей программы, информация в файл запишется в результате вызова функция \texttt{fclose()};
	\item из-за того \texttt{f\_pos} независимы для каждого дескриптора файла, запись в файл будет производится с самого начала;
	\item таким образом, информация записаная при первом вызове \texttt{fclose()} будет потеряна в результате второго вызова \texttt{fclose()} (см. рис. \ref{fig:prog_03}).
	\item в многопоточной реализации результат аналогичен - с помощью \texttt{pthread\_join} мы дожидаемся вызова \texttt{fclose()} для \texttt{f2} в отдельном потоке и далее вызываем \texttt{fclose()} для \texttt{f1}.
\end{itemize}

\begin{lstlisting}[language=c, label=p3, caption=Программа №3]
#include <stdio.h>
#include <fcntl.h>
#include <unistd.h>
#define OK 0
#define FILE_NAME "data/out.txt"
#define SPEC "%c"

int main()
{
	FILE *f1 = fopen(FILE_NAME, "w");
	FILE *f2 = fopen(FILE_NAME, "w");
	
	for (char c = 'a'; c <= 'z'; c++)
	{
		if (c % 2)
		{
			fprintf(f1, SPEC, c);
		}
		else
		{
			fprintf(f2, SPEC, c);
		}
	}
	
	fclose(f2);
	fclose(f1);
	return OK;
}
\end{lstlisting}

""\newline
\textbf{На рис. \ref{fig:prog_03} представлен результат работы третьей программы.}

\begin{figure}[H]
	\centering
	\includegraphics[scale=0.8]{img/prog_03.png}
	\caption{Результат работы третьей программы}
	\label{fig:prog_03}
\end{figure}

\begin{lstlisting}[language=c, label=p3thread, caption=Программа №3 (реализация с потоками)]
#include <stdio.h>
#include <fcntl.h>
#include <pthread.h>
#include <unistd.h>
#define OK 0
#define FILE_NAME "data/out.txt"
#define SPEC "%c"

void *run_buffer(void *args)
{
	FILE *f = (FILE *)args;
	
	for (char c = 'b'; c <= 'z'; c += 2)
	{
		fprintf(f, SPEC, c);
	}
	
	fclose(f);
	return NULL;
}

int main()
{
	FILE *f1 = fopen(FILE_NAME, "w");
	FILE *f2 = fopen(FILE_NAME, "w");
	
	pthread_t thread;
	int rc = pthread_create(&thread, NULL, run_buffer, (void *)(f2));
	
	for (char c = 'a'; c <= 'z'; c += 2)
	{
		fprintf(f1, SPEC, c);
	}
	
	pthread_join(thread, NULL);
	fclose(f1);
	return OK;
}
\end{lstlisting}

""\newline
\textbf{На рис. \ref{fig:prog_03_thread} представлен результат работы второй программы (с потоками).}

\begin{figure}[H]
	\centering
	\includegraphics[scale=0.8]{img/prog_03_thread.png}
	\caption{Результат работы третьей программы (с потоками)}
	\label{fig:prog_03_thread}
\end{figure}

\bibliographystyle{utf8gost705u}  % стилевой файл для оформления по ГОСТу
\bibliography{51-biblio}          % имя библиографической базы (bib-файла)
	
\end{document}
